\documentclass[a4paper,12pt,twoside]{report}

\usepackage{seanodes_spec}
%\usepackage{subfig}

\graphicspath{{./images/}}

\title{Exanodes' Architecture}
\subtitle{High level overview}
\author{Damien Collard}
\activity{Core}
\project{Architecture}

\begin{document}

\makesimpleheader

\section{\label{sec:introduction}Introduction}

This document presents a (very) high level overview of Exanodes'
current architecture: its layers and processes, as well as the control
and data paths.

The current architecture is crossplatform: it is the same on both
Linux and Windows, \ie logical entities and OS entities (processes)
are exactly the same.

To achieve portability, Exanodes does two things:
\begin{compactitem}
\item It uses a library called ``OS lib'', which abstracts OS-specific
  details and present a common interface (for processes,
  synchronization primitives, etc).
\item It is built with CMake, which exists on both OSes.
\end{compactitem}

Note that the OS library doesn't appear in the architecture diagrams
because it's just a {\em tool}, not an entity per se.

\section{\label{sec:architecture}Architecture}

An instance of the Exanodes software runs on each node whose storage
is to be shared.

Exanodes uses two paths (which may be routed through the same physical
medium or different ones): the control path is used for internal
cluster control and the data path is used to transfer the data to be
stored/retrieved. The control and data paths are shown in
\prettyref{fig:control_and_data_paths}.

\begin{figure}[ht]
  \centering
  \includegraphics[width=0.6\textwidth]{exanodes_control_and_data_paths.eps}
  \caption{\label{fig:control_and_data_paths}Control and data paths.}
\end{figure}

This figure shows two nodes. On each node, the layers comprising
Exanodes, also called ``services'', are shown with their interactions.

The layers are as follows:
\begin{itemize}
\item Csupd: the cluster supervision layer \cite{csupd, mship}. Each
  instance of Csupd is a failure detector. The role of this layer is
  to determine what are the nodes comprising a cluster at any point in
  time.
\item Admind: the administration layer. It receives user commands
  \cite{cli_server} and directs all Exanodes operations. It takes
  decisions based on Csupd's input \cite{evmgr, mship}.
\item NBD: the network block device \cite{nbd}. This layer is in charge of
  transmitting the (user) data.
\item VRT: the virtualizer \cite{vrt}, whose role is to handle data
  placement and offers different reliability levels (simple striping,
  rainX).
\item FS: the filesystem layer, which allows integration with
  filesystems, either standard (Ext3) or distributed (GFS).
\end{itemize}

An Exanodes instance comprises several daemons, as shown in
\prettyref{fig:daemons}. Most daemons have a single role, but some of
them (for performance reasons) have several roles. For instance, the
``client'' NBD daemon also contains the VRT because of their strong
interactions on the data path.

\begin{figure}[ht]
  \centering
  \includegraphics[width=0.2\textwidth]{exanodes_daemons.eps}
  \caption{\label{fig:daemons}Daemons.}
\end{figure}

On each node, the Admind daemon contains a stack of service instances
(a service instance is the intersection of a service/layer and a
node). \prettyref{fig:services_and_executives} shows the service
stack. In case of node or disk failure, Exanodes performs a recovery
by walking the service stack. Each service conforms to the recovery
framework and is responsible for the recovery of its own data and
metadata.

\begin{figure}[ht]
  \centering
  \includegraphics[width=0.6\textwidth]{exanodes_services_and_executives.eps}
  \caption{\label{fig:services_and_executives}Services and executives.}
\end{figure}

\prettyref{fig:request_queues} shows a simplified view of the data
request path from the target to the NBD \cite{request_path}. Note that
a ``target'' here is either an actual iSCSI target (Windows, Linux) or block
device (Linux).

\begin{figure}[ht]
  \centering
  \includegraphics[width=0.6\textwidth]{request_queues_client_side.eps}
  \caption{\label{fig:request_queues}Data path: request queues from
    the target(s) to the NBD.}
\end{figure}

% FIXME Use figure 1(b) from http://builder/builds/pdf/activity/innovation/project/perf_study/measures_spec/measures_spec.pdf ?

% FIXME Use figure 1 from http://builder/builds/pdf/activity/innovation/project/quota/stream_model/quota_stream_model.pdf ?

\newpage
\bibliography{exanodes_arch.bib}

\end{document}
